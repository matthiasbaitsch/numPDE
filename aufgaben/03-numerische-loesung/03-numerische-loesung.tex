% Options for packages loaded elsewhere
% Options for packages loaded elsewhere
\PassOptionsToPackage{unicode}{hyperref}
\PassOptionsToPackage{hyphens}{url}
\PassOptionsToPackage{dvipsnames,svgnames,x11names}{xcolor}
%
\documentclass[
  ngerman,
  letterpaper,
  DIV=11]{scrartcl}
\usepackage{xcolor}
\usepackage{amsmath,amssymb}
\setcounter{secnumdepth}{2}
\usepackage{iftex}
\ifPDFTeX
  \usepackage[T1]{fontenc}
  \usepackage[utf8]{inputenc}
  \usepackage{textcomp} % provide euro and other symbols
\else % if luatex or xetex
  \usepackage{unicode-math} % this also loads fontspec
  \defaultfontfeatures{Scale=MatchLowercase}
  \defaultfontfeatures[\rmfamily]{Ligatures=TeX,Scale=1}
\fi
\usepackage{lmodern}
\ifPDFTeX\else
  % xetex/luatex font selection
\fi
% Use upquote if available, for straight quotes in verbatim environments
\IfFileExists{upquote.sty}{\usepackage{upquote}}{}
\IfFileExists{microtype.sty}{% use microtype if available
  \usepackage[]{microtype}
  \UseMicrotypeSet[protrusion]{basicmath} % disable protrusion for tt fonts
}{}
\makeatletter
\@ifundefined{KOMAClassName}{% if non-KOMA class
  \IfFileExists{parskip.sty}{%
    \usepackage{parskip}
  }{% else
    \setlength{\parindent}{0pt}
    \setlength{\parskip}{6pt plus 2pt minus 1pt}}
}{% if KOMA class
  \KOMAoptions{parskip=half}}
\makeatother
% Make \paragraph and \subparagraph free-standing
\makeatletter
\ifx\paragraph\undefined\else
  \let\oldparagraph\paragraph
  \renewcommand{\paragraph}{
    \@ifstar
      \xxxParagraphStar
      \xxxParagraphNoStar
  }
  \newcommand{\xxxParagraphStar}[1]{\oldparagraph*{#1}\mbox{}}
  \newcommand{\xxxParagraphNoStar}[1]{\oldparagraph{#1}\mbox{}}
\fi
\ifx\subparagraph\undefined\else
  \let\oldsubparagraph\subparagraph
  \renewcommand{\subparagraph}{
    \@ifstar
      \xxxSubParagraphStar
      \xxxSubParagraphNoStar
  }
  \newcommand{\xxxSubParagraphStar}[1]{\oldsubparagraph*{#1}\mbox{}}
  \newcommand{\xxxSubParagraphNoStar}[1]{\oldsubparagraph{#1}\mbox{}}
\fi
\makeatother

\usepackage{color}
\usepackage{fancyvrb}
\newcommand{\VerbBar}{|}
\newcommand{\VERB}{\Verb[commandchars=\\\{\}]}
\DefineVerbatimEnvironment{Highlighting}{Verbatim}{commandchars=\\\{\}}
% Add ',fontsize=\small' for more characters per line
\usepackage{framed}
\definecolor{shadecolor}{RGB}{241,243,245}
\newenvironment{Shaded}{\begin{snugshade}}{\end{snugshade}}
\newcommand{\AlertTok}[1]{\textcolor[rgb]{0.68,0.00,0.00}{#1}}
\newcommand{\AnnotationTok}[1]{\textcolor[rgb]{0.37,0.37,0.37}{#1}}
\newcommand{\AttributeTok}[1]{\textcolor[rgb]{0.40,0.45,0.13}{#1}}
\newcommand{\BaseNTok}[1]{\textcolor[rgb]{0.68,0.00,0.00}{#1}}
\newcommand{\BuiltInTok}[1]{\textcolor[rgb]{0.00,0.23,0.31}{#1}}
\newcommand{\CharTok}[1]{\textcolor[rgb]{0.13,0.47,0.30}{#1}}
\newcommand{\CommentTok}[1]{\textcolor[rgb]{0.37,0.37,0.37}{#1}}
\newcommand{\CommentVarTok}[1]{\textcolor[rgb]{0.37,0.37,0.37}{\textit{#1}}}
\newcommand{\ConstantTok}[1]{\textcolor[rgb]{0.56,0.35,0.01}{#1}}
\newcommand{\ControlFlowTok}[1]{\textcolor[rgb]{0.00,0.23,0.31}{\textbf{#1}}}
\newcommand{\DataTypeTok}[1]{\textcolor[rgb]{0.68,0.00,0.00}{#1}}
\newcommand{\DecValTok}[1]{\textcolor[rgb]{0.68,0.00,0.00}{#1}}
\newcommand{\DocumentationTok}[1]{\textcolor[rgb]{0.37,0.37,0.37}{\textit{#1}}}
\newcommand{\ErrorTok}[1]{\textcolor[rgb]{0.68,0.00,0.00}{#1}}
\newcommand{\ExtensionTok}[1]{\textcolor[rgb]{0.00,0.23,0.31}{#1}}
\newcommand{\FloatTok}[1]{\textcolor[rgb]{0.68,0.00,0.00}{#1}}
\newcommand{\FunctionTok}[1]{\textcolor[rgb]{0.28,0.35,0.67}{#1}}
\newcommand{\ImportTok}[1]{\textcolor[rgb]{0.00,0.46,0.62}{#1}}
\newcommand{\InformationTok}[1]{\textcolor[rgb]{0.37,0.37,0.37}{#1}}
\newcommand{\KeywordTok}[1]{\textcolor[rgb]{0.00,0.23,0.31}{\textbf{#1}}}
\newcommand{\NormalTok}[1]{\textcolor[rgb]{0.00,0.23,0.31}{#1}}
\newcommand{\OperatorTok}[1]{\textcolor[rgb]{0.37,0.37,0.37}{#1}}
\newcommand{\OtherTok}[1]{\textcolor[rgb]{0.00,0.23,0.31}{#1}}
\newcommand{\PreprocessorTok}[1]{\textcolor[rgb]{0.68,0.00,0.00}{#1}}
\newcommand{\RegionMarkerTok}[1]{\textcolor[rgb]{0.00,0.23,0.31}{#1}}
\newcommand{\SpecialCharTok}[1]{\textcolor[rgb]{0.37,0.37,0.37}{#1}}
\newcommand{\SpecialStringTok}[1]{\textcolor[rgb]{0.13,0.47,0.30}{#1}}
\newcommand{\StringTok}[1]{\textcolor[rgb]{0.13,0.47,0.30}{#1}}
\newcommand{\VariableTok}[1]{\textcolor[rgb]{0.07,0.07,0.07}{#1}}
\newcommand{\VerbatimStringTok}[1]{\textcolor[rgb]{0.13,0.47,0.30}{#1}}
\newcommand{\WarningTok}[1]{\textcolor[rgb]{0.37,0.37,0.37}{\textit{#1}}}

\usepackage{longtable,booktabs,array}
\usepackage{calc} % for calculating minipage widths
% Correct order of tables after \paragraph or \subparagraph
\usepackage{etoolbox}
\makeatletter
\patchcmd\longtable{\par}{\if@noskipsec\mbox{}\fi\par}{}{}
\makeatother
% Allow footnotes in longtable head/foot
\IfFileExists{footnotehyper.sty}{\usepackage{footnotehyper}}{\usepackage{footnote}}
\makesavenoteenv{longtable}
\usepackage{graphicx}
\makeatletter
\newsavebox\pandoc@box
\newcommand*\pandocbounded[1]{% scales image to fit in text height/width
  \sbox\pandoc@box{#1}%
  \Gscale@div\@tempa{\textheight}{\dimexpr\ht\pandoc@box+\dp\pandoc@box\relax}%
  \Gscale@div\@tempb{\linewidth}{\wd\pandoc@box}%
  \ifdim\@tempb\p@<\@tempa\p@\let\@tempa\@tempb\fi% select the smaller of both
  \ifdim\@tempa\p@<\p@\scalebox{\@tempa}{\usebox\pandoc@box}%
  \else\usebox{\pandoc@box}%
  \fi%
}
% Set default figure placement to htbp
\def\fps@figure{htbp}
\makeatother



\ifLuaTeX
\usepackage[bidi=basic]{babel}
\else
\usepackage[bidi=default]{babel}
\fi
% get rid of language-specific shorthands (see #6817):
\let\LanguageShortHands\languageshorthands
\def\languageshorthands#1{}
\ifLuaTeX
  \usepackage[german]{selnolig} % disable illegal ligatures
\fi


\setlength{\emergencystretch}{3em} % prevent overfull lines

\providecommand{\tightlist}{%
  \setlength{\itemsep}{0pt}\setlength{\parskip}{0pt}}



 


\KOMAoption{captions}{tableheading}
\makeatletter
\@ifpackageloaded{caption}{}{\usepackage{caption}}
\AtBeginDocument{%
\ifdefined\contentsname
  \renewcommand*\contentsname{Inhaltsverzeichnis}
\else
  \newcommand\contentsname{Inhaltsverzeichnis}
\fi
\ifdefined\listfigurename
  \renewcommand*\listfigurename{Abbildungsverzeichnis}
\else
  \newcommand\listfigurename{Abbildungsverzeichnis}
\fi
\ifdefined\listtablename
  \renewcommand*\listtablename{Tabellenverzeichnis}
\else
  \newcommand\listtablename{Tabellenverzeichnis}
\fi
\ifdefined\figurename
  \renewcommand*\figurename{Abbildung}
\else
  \newcommand\figurename{Abbildung}
\fi
\ifdefined\tablename
  \renewcommand*\tablename{Tabelle}
\else
  \newcommand\tablename{Tabelle}
\fi
}
\@ifpackageloaded{float}{}{\usepackage{float}}
\floatstyle{ruled}
\@ifundefined{c@chapter}{\newfloat{codelisting}{h}{lop}}{\newfloat{codelisting}{h}{lop}[chapter]}
\floatname{codelisting}{Listing}
\newcommand*\listoflistings{\listof{codelisting}{Listingverzeichnis}}
\makeatother
\makeatletter
\makeatother
\makeatletter
\@ifpackageloaded{caption}{}{\usepackage{caption}}
\@ifpackageloaded{subcaption}{}{\usepackage{subcaption}}
\makeatother
\usepackage{bookmark}
\IfFileExists{xurl.sty}{\usepackage{xurl}}{} % add URL line breaks if available
\urlstyle{same}
\hypersetup{
  pdftitle={Aufgabenblatt 3},
  pdflang={de},
  colorlinks=true,
  linkcolor={blue},
  filecolor={Maroon},
  citecolor={Blue},
  urlcolor={Blue},
  pdfcreator={LaTeX via pandoc}}


\title{Aufgabenblatt 3}
\usepackage{etoolbox}
\makeatletter
\providecommand{\subtitle}[1]{% add subtitle to \maketitle
  \apptocmd{\@title}{\par {\large #1 \par}}{}{}
}
\makeatother
\subtitle{Numerische Lösung}
\author{}
\date{}
\begin{document}
\maketitle


\newcommand{\SI}[2]{#1\,#2}
\newcommand{\si}[1]{#1}
\newcommand{\num}[1]{#1}
\newcommand{\newton}{\mathrm{N}}
\newcommand{\metre}{\mathrm{m}}
\newcommand{\per}{/}
\newcommand{\squared}{^2}
\newcommand{\kilo}{\mathrm{k}}
\newcommand{\gram}{\mathrm{g}}
\newcommand{\second}{\mathrm{s}}
\newcommand{\tothe}[1]{^#1}

Wir lösen das Bohrpfahlproblem mit der Finite-Elemente-Methode für die
folgenden Systemparameter:

\[
    \begin{split}
      E & = 35 \cdot 10^9\,\mathrm{N}/\mathrm{m}^2
      &
      d & = 0.8\,\mathrm{m}
      &
      l & = 20\,\mathrm{m}
      \\[0.5em]
      F & = 2 \cdot 10^6\,\mathrm{N}
      &
      \rho & = 2500\,\mathrm{k}\mathrm{g}/\mathrm{m}^3\qquad
      &
      g & = 9.81\,\mathrm{m}/\mathrm{s}^2
      \\[0.5em]
      C & = 1.75 \cdot 10^7 \, \pi \, d \, \mathrm{N}/\mathrm{m}^2\qquad
      &
      S & = 120 \cdot 10^6\,\mathrm{N}/\mathrm{m}
    \end{split}
\]

\section{Notebook einrichten}\label{notebook-einrichten}

Legen Sie ein Jupyter-Notebook für dieses Paket an und laden sie die
folgenden Bibliotheken.

\begin{Shaded}
\begin{Highlighting}[]
\ImportTok{using} \BuiltInTok{DotMaps}
\ImportTok{using} \BuiltInTok{CairoMakie}
\ImportTok{using} \BuiltInTok{MMJMesh.Plots}
\ImportTok{using} \BuiltInTok{MMJMesh.Mathematics}
\end{Highlighting}
\end{Shaded}

\section{Problemparameter}\label{problemparameter}

Für die Programmierung bietet es sich an, die Systemparameter in einem
Struct zu speichern. Das geht so:

\begin{Shaded}
\begin{Highlighting}[]
\NormalTok{p }\OperatorTok{=} \FunctionTok{DotMap}\NormalTok{()}

\CommentTok{\# Basiswerte}
\NormalTok{p.E }\OperatorTok{=} \FloatTok{35e9}
\NormalTok{p.F }\OperatorTok{=} \FloatTok{2e6}
\OperatorTok{...}

\CommentTok{\# Abgeleitete Werte}
\NormalTok{p.EA }\OperatorTok{=} 
\OperatorTok{...}
\NormalTok{p.Ω }\OperatorTok{=} \FloatTok{0} \OperatorTok{..}\NormalTok{ p.l}
\end{Highlighting}
\end{Shaded}

Unter einem Struct kann man sich eine Mehrfachvariable vorstellen, die
unter einem übergeordnetem Namen mehrere Werte speichert. Auf die Werte
in dem Struct kann dann mit dem Punkt zugegriffen werden, so ist zum
Beispiel \texttt{p.E} der E-Modul.

\section{Exakte Lösung}\label{exakte-luxf6sung}

Für den Bohrpfahl lässt sich die Lösungsfunktion analytisch bestimmen,
zum Beispiel mithilfe des Computeralgebrasystems Mathematica. Für die
Parameter oben lautet die Verschiebungsfunktion

\[
\begin{multlined}
  u(x) 
  = 
  \frac
  {\exp(-x/20)}
  {3500000 \cdot \pi \cdot  \left(a_1 \cdot e^2 - a_2\right)}
  \cdot
  \bigg(
    25000 \cdot \big(a_1 \cdot e^2 + a_2 \cdot \exp(x / 10)\big)
    \\[1em]
    - 2943 \cdot \pi \cdot e \cdot \big(\exp(x / 10) + 1\big)
    \\[1em]
    + 981 \cdot \pi \cdot \exp(x / 20) \cdot \big(a_1 \cdot e^2 - a_2\big)
  \bigg).
\end{multlined}
\]

wobei \(a_1 = 3 + 7 \pi\) und \(a_2 = 7\pi - 3\) gelten soll. Die
Eulersche Zahl \(e\) geben Sie mit \texttt{\textbackslash{}euler\ Tab}
ein.

Geben Sie die exakte Lösung in ihr Notebook ein. Sie können den
folgenden Code als Ausgangspunkt verwenden:

\begin{Shaded}
\begin{Highlighting}[]
\NormalTok{a1 }\OperatorTok{=} \FloatTok{99}
\NormalTok{a2 }\OperatorTok{=} \FloatTok{99}
\NormalTok{x }\OperatorTok{=} \FunctionTok{parameter}\NormalTok{(p.Ω)}

\NormalTok{u }\OperatorTok{=} \FunctionTok{exp}\NormalTok{(x }\OperatorTok{/} \FloatTok{20}\NormalTok{) }\OperatorTok{+} \FunctionTok{exp}\NormalTok{(}\OperatorTok{{-}}\NormalTok{x }\OperatorTok{/} \FloatTok{10}\NormalTok{)}

\FunctionTok{fplot}\NormalTok{(u, npoints}\OperatorTok{=}\FloatTok{100}\NormalTok{)}
\end{Highlighting}
\end{Shaded}

Plotten Sie zusätzlich die Normalkraft \(N = EA \cdot u'\).

\section{Kontrolle starke Form}\label{kontrolle-starke-form}

Kontrollieren Sie, ob die Funktion \(u\) wirklich das Randwertproblem
löst. Plotten Sie hierzu das Residuum \(EA u'' - C u + n\) und
überprüfen Sie, ob die Randbedingungen erfüllt sind.

\section{Schwache Form und Kontrolle}\label{schwache-form-und-kontrolle}

Geben Sie die Bilinear- und Linearform der schwachen Formulierung des
Problems ein. Hier eine Idee, wie das funktioniert:

\begin{Shaded}
\begin{Highlighting}[]
\FunctionTok{a}\NormalTok{(u, δu) }\OperatorTok{=} \FunctionTok{integrate}\NormalTok{(u}\OperatorTok{\textquotesingle{}} \OperatorTok{*}\NormalTok{ δu}\OperatorTok{\textquotesingle{}}\NormalTok{, p.Ω)}
\end{Highlighting}
\end{Shaded}

Überprüfen Sie für die Funktionen

\begin{Shaded}
\begin{Highlighting}[]
\NormalTok{x }\OperatorTok{=} \FunctionTok{parameter}\NormalTok{(p.Ω)}
\NormalTok{δu1 }\OperatorTok{=} \FunctionTok{sin}\NormalTok{(x)}
\NormalTok{δu2 }\OperatorTok{=}\NormalTok{ x}
\NormalTok{δu3 }\OperatorTok{=} \FunctionTok{Polynomial}\NormalTok{([}\FloatTok{0}\NormalTok{, }\FloatTok{0}\NormalTok{, }\FloatTok{1}\NormalTok{])}
\end{Highlighting}
\end{Shaded}

ob die Gleichung der Schwachen Form erfüllt ist. Es bietet sich an,
hierzu die prozentuale Abweichung

\[
  100 \cdot \frac{|a(u, \delta u) - b(\delta u)|}{|a(u, \delta u)|}
\]

zu betrachten.

\section{Numerische Lösung}\label{numerische-luxf6sung}

Erstellen Sie eine Funktion \texttt{pile\_fem(p,\ k)}, die für gegebene
Problemparameter \texttt{p} und eine Anzahl von Elementen \texttt{k} die
Näherungslösung \(u_h\) berechnet. Verwenden Sie den folgenden Code als
Ausgangspunkt:

\begin{Shaded}
\begin{Highlighting}[]
\KeywordTok{function} \FunctionTok{pile\_fem}\NormalTok{(p, k)}
\NormalTok{    phis }\OperatorTok{=} \FunctionTok{hatfunctions}\NormalTok{(}\FunctionTok{range}\NormalTok{(p.Ω, k }\OperatorTok{+} \FloatTok{1}\NormalTok{))}
    \CommentTok{\# Hier die Berechnung}
\NormalTok{    uHat }\OperatorTok{=} \FunctionTok{rand}\NormalTok{(k }\OperatorTok{+} \FloatTok{1}\NormalTok{)}
    \ControlFlowTok{return} \FunctionTok{sum}\NormalTok{(uHat }\OperatorTok{.*}\NormalTok{ phis)}
\KeywordTok{end}

\NormalTok{uh }\OperatorTok{=} \FunctionTok{pile\_fem}\NormalTok{(p, }\FloatTok{3}\NormalTok{)}
\FunctionTok{fplot}\NormalTok{(uh, u, npoints}\OperatorTok{=}\FloatTok{100}\NormalTok{)}
\end{Highlighting}
\end{Shaded}

Plotten Sie darüber hinaus die Näherung für die Normalkraft \(N_h\)
sowie den Fehler \(e = u - u_h\).

\section{Konvergenz}\label{konvergenz}

\begin{enumerate}
\def\labelenumi{\arabic{enumi}.}
\item
  Fehlerfunktion: Plotten Sie die Fehlerfunktion \(e = u - u_h\) für
  \(k = 2, 4, 8, \dots, 2^5\) Elemente.

  Tipp:

\begin{Shaded}
\begin{Highlighting}[]
\FloatTok{2} \OperatorTok{.\^{}}\NormalTok{ (}\FloatTok{1}\OperatorTok{:}\FloatTok{5}\NormalTok{)}
\end{Highlighting}
\end{Shaded}
\item
  Normalkraft: Stellen Sie in einem Plot die approximierte Normalkraft
  \(N_h\) für \(k = 2, 4, 8, \dots, 2^5\) Elemente zusammen mit der
  exakten Lösung \(N\) dar. Wie erklären Sie sich den Verlauf?

  Tipp: Eine besser lesbare Darstellung erhalten Sie mit

\begin{Shaded}
\begin{Highlighting}[]
\FunctionTok{fplot!}\NormalTok{(Nh, connect\_jumps}\OperatorTok{=}\ConstantTok{true}\NormalTok{)}
\end{Highlighting}
\end{Shaded}
\item
  Fehler in max-Norm: Berechnen und plotten Sie den prozentualen Fehler

  \[
   100 \cdot \frac{\|u - u_h\|_\mathrm{max}}{\|u\|_\mathrm{max}}
   \]

  für \(k = 2, 4, 8, \dots, 2^5\) Elemente.
\end{enumerate}




\end{document}
